\documentclass[12pt,a4paper,openright]{book}
\usepackage[utf8]{inputenc}
\usepackage[spanish]{babel}
\usepackage{amsmath}
\usepackage{amsfonts}
\usepackage{amssymb}
\usepackage{graphicx}
\author{Santiago de Diego\\Braulio Valdivielso\\Francisco Luque}
\title{Probabilidad I}
\date{}
\begin{document}
\maketitle
\newtheorem{theorem}{Teorema}[section]
\newtheorem{lemma}{Lema}[section]
\newtheorem{proof}{Demostración}[section]
\newpage
\chapter*{Agradecimientos} % si no queremos que añada la palabra "Capitulo"
\addcontentsline{toc}{chapter}{Agradecimientos} % si queremos que aparezca en el índice
\markboth{AGRADECIMIENTOS}{AGRADECIMIENTOS} % encabezado 
 
¡Muchas gracias a todos!yo

\chapter*{Resumen} % si no queremos que añada la palabra "Capitulo"
\addcontentsline{toc}{chapter}{Resumen} % si queremos que aparezca en el índice
\markboth{RESUMEN}{RESUMEN} % encabezado
%Resumen

\tableofcontents
\chapter{Conjuntos y funciones $\sigma$-aditivas}
\section{Introducción: concepto de $\sigma$ aditividad}
Sea un conjunto $\Omega$ y una clase de conjuntos $\mathbb{C}$ sobre $\Omega$. Definimos:
$$\varphi : \mathbb{C}\rightarrow \mathbb{R}$$
$\varphi (s)$ es único\\\\
Se dice que $\varphi$ es aditiva si $\varphi(\displaystyle\sum_{1}^{n}A_k)=\displaystyle\sum_1^n\mu(A_k)$\\
Se dice que $\varphi$ es $\sigma$ es aditiva si $\varphi(\displaystyle\sum_1^\infty A_k)=\displaystyle\sum_1^\infty \mu(A_k)$\\\\
\begin{lemma} 
Si $\exists B\, : \, \varphi(B)<\infty \Rightarrow \varphi(\emptyset)=0$\\\\
\end{lemma}

\begin{theorem}
Si $\varphi$ es $\sigma$ aditiva y $\displaystyle\sum\varphi(A_i)<\infty\Rightarrow\displaystyle\sum(\vert \varphi(A_i)\vert)<\infty$\\\\
\end{theorem}
\begin{proof}
Tomemos las sucesiones: \\\\
Si $\varphi(A_n)\geq 0 \Longrightarrow A_n^{+}=A_n$ y $A_n^{-}=\emptyset$\\
Si $\varphi(A_n)<0 \Longrightarrow A_n^{+}=\emptyset$ y $A_n^{-}=A_n$\\
Entonces $\varphi(\displaystyle\sum A_n^{+})=\displaystyle\sum\varphi(A_n^{+})$ y $\varphi(\displaystyle\sum A_n^{-})=\displaystyle\sum\varphi(A_n^{-})$\\\\
\end{proof}

\begin{theorem} 
Si $\varphi$ es $\sigma$ aditiva, $\varphi\geq 0$, entonces: $\varphi$ es no decreciente y subaditiva
\end{theorem}

\section{Límites en sucesiones de conjuntos}
\subsection{Límites en sucesiones monótonas}
Si tomamos la relación de inclusión entre conjuntos ($\subseteq$) como una relación de orden, podemos hablar sin ningún tipo de problema de sucesiones monótonas. En este sentido, una sucesión creciente de conjuntos sería una sucesión $\{A_n\}_{n \in \mathbb{N}}$ en la que se cumple que $A_i \subseteq A_{i+1} \forall i \in \mathbb{N}$. De forma análoga se podría ver qué es una sucesión decreciente de conjuntos.

Resulta que existe una forma intuitiva de definir el límite de una sucesión monótona de conjuntos. En particular, el límite de una sucesión creciente de conjuntos $\{A_n\}_{n \in \mathbb{N} }$ se puede definir como 


$$\lim_{n\to\infty} A_n = \bigcap_{k \in \mathbb{N}}^{\infty} A_k$$


Esta definición aprovecha la relación de orden entre los conjuntos de la sucesión para afirmar que \textit{si un elemento está en el último conjunto de la sucesión, entonces está en todos los anteriores, y por tanto en \textbf{todos}}, por ello se puede utilizar una intersección numerable (que está perfectamente definida) para formalizar el concepto.

Con una intuición análoga se define el límite de una sucesión decreciente de conjuntos. Si $\{A_n\}_{n\in\mathbb{N}}$ es una sucesión decreciente de conjuntos, entonces:
$$ \lim_{n\to\infty} A_n = \bigcup_{k\in\mathbb{N}}^{\infty} A_k $$

\subsection{Límites superiores e inferiores}
No solo se puede hablar de límites en sucesiones monótonas. Para definir los límites en sucesiones arbitrarias de conjuntos tenemos que recurrir a los conceptos de límite inferior y límite superior. La intuición de estos límites superior e inferior pasan por el concepto de las colas de la sucesión.

Dada una sucesión de conjuntos $\{A_n\}_{n\in\mathbb{N} }$ podemos considerar el conjunto
$$ B_n = \bigcap_{k=n}^{\infty} A_k $$ 
 y este conjunto contiene aquellos elementos que están en \textbf{todos} los $A_k$ para $k \geq n$. Es fácil probar que la sucesión $\{B_n\}_{n\in\mathbb{N}}$ es una sucesión creciente de conjuntos, y por tanto se puede obtener el límite $\lim_{n\to\infty} B_n$. Informalmente, ese límite es un conjunto que contiene a todos los elementos de $A_n$ que están en todos los conjuntos $A_k$ a partir de cierto $n\in\mathbb{N}$. Definimos el límite inferior de $A_n$ como

 $$ \liminf A_n = \lim_{n\to\infty} B_n = \bigcup_{n\in\mathbb{N}} \bigcap_{k=n}^{\infty} A_n $$

 Análogamente, podríamos definir la sucesión de conjuntos $\{C_n\}_{n\in\mathbb{N}}$ como

 $$ C_n = \bigcup_{k=n}^{\infty} A_n $$

Cada $C_n$ contiene todos los elementos que están presentes en algún $A_k$ para $k \geq n$. Es fácil también ver que la sucesión $\{C_n\}_{n\in\mathbb{N}}$ es una sucesión decreciente de conjuntos, y por tanto su límite también está bien definido. Se puede definir entonces el límite superior de $\{A_n\}_{n\in\mathbb{N}}$
como 

$$ \limsup A_n = \lim_{n\to\infty} C_n = \bigcap_{n\in\mathbb{N}} \bigcup_{k=n}^{\infty} A_n $$

Informalmente se puede pensar en este límite superior de $A_n$ como el conjunto de los elementos que están en infinitos conjuntos de la sucesión.

A partir de estas definiciones, es fácil comprobar que 
$$ \liminf A_n \subseteq \limsup A_n $$

\subsection{Límite de una sucesión de conjuntos}
Diremos que una sucesión de conjuntos tiene límite si su límite inferior y superior coinciden, y el límite tendrá como valor efectivamente el de estos límites. Es decir: 
$$ \lim A_n = \liminf A_n = \limsup A_n $$ 
en caso de que límite superior e inferior coincidan.

\section{Espacios de probabilidad y variables aleatorias}
\begin{theorem} (de extensión de Caratheodory) 
Una medida $\mu$ sobre $\mathbb{C}$ se extiende a $A(\mathbb{C})$. La extensión es mínima si $\mu$ es finita.
$$(\Omega,\mathbb{C},\mu)\overbrace{\longrightarrow}^{extension}(\Omega,S(\Omega),\mu^\circ)\overbrace{\longrightarrow}^{restriccion}(\Omega,A^\circ,\mu^\circ)\overbrace{\longrightarrow}^{restriccion}(\Omega,A(\mathbb{C}),\mu^\circ)$$
donde:\\\\
$\mu^\circ$ es la medida exterior que es extensión de $\mu$\\
$A^\circ$ es el $\sigma$-campo\\\\
$\mu^\circ$ es medida exterior si es subaditiva, no decreciente y $\mu^\circ(\emptyset)=0$
\\\\
$A\in \Omega$, $\mu^\circ(A)=inf\displaystyle\sum(A_j)$\\\\
$\mu^\circ(\emptyset)\leq\mu(\emptyset)$\\\\
$a\in A^\circ$ si $D\in S(\Omega)$ se cumple que $\mu^\circ(D)\geq\mu^\circ(AD)+\mu^\circ(A^c D)$
\end{theorem}

\begin{lemma} 
$\mathbb{X}(w)=\varphi(w)\,\forall w \in \Omega - \Lambda\, t.q.\, P(\Lambda)=0$. Entonces, $\mathbb{X}=\varphi$ \\\\
$P(\mathbb{X}=\varphi)=1$\\\\
\end{lemma}
Definimos la relación de equivalencia $\mathbb{X}R\varphi\leftrightarrow P(\mathbb{X}=\varphi)=1$
\section{Distribuciones de probabilidad}
\section{Funciones de distribución}
\section{Convergencia en probabilidad de espacios métricos}
\textbf{Definición de convergencia en probabilidad:} Si $P(| \mathbb{X}_n- \mathbb{X} | \geq\epsilon)\rightarrow 1$, entonces se dice que $\mathbb{X}_n\overbrace{\longrightarrow}^P \mathbb{X}$, es decir, $\mathbb{X}_n$ converge en probabilidad a $\mathbb{X}$
\\\\
\textbf{Lema:} $\mathbb{X}_n\overbrace{\longrightarrow}^P \mathbb{X} \bigwedge \mathbb{X}_n\overbrace{\longrightarrow}^P \mathbb{\varphi}\Longrightarrow P(\mathbb{X}=\varphi)=1$\\\\
\textit{\textbf{Demostración}}\\\\
$| \mathbb{X}-\varphi |=|-\mathbb{X}_n - \mathbb{X}-\varphi + \mathbb{X}_n | \leq |\mathbb{X}_n - \mathbb{X}|+|\mathbb{X}_n-\varphi |$\\\\
$P(| \mathbb{X}- \varphi | \geq\epsilon)\leq P(| \mathbb{X}_n- \mathbb{X} | \geq\frac{\epsilon}{2})+P(| \mathbb{X}_n- \varphi | \leq\frac{\epsilon}{2})\rightarrow 0$
\begin{flushright}
$\square$
\end{flushright}
\textbf{Teorema:} Una sucesión $\mathbb{X}_n$ converge en probabilidad a $\mathbb{X}$ si y solo si:
$$P(| \mathbb{X}_{n+\delta}-\mathbb{X}_n |\geq\epsilon)\longrightarrow 0$$
\textit{\textbf{Demostración}}\\\\
 $| \mathbb{X}_{n+\delta}-\mathbb{X}_n |= |   \mathbb{X}_{n+\delta}-\mathbb{X}+\mathbb{X}-\mathbb{X}_n  |$\\\\
 Y se termina aplicando la desigualdad de Cauchy como antes.
 \begin{flushright}
$\square$
\end{flushright}
En el siguiente ejemplo consideramos una sucesión de variables aleatorias indicadoras tal que:\\\\
$P(\mathbb{X}_n=1)=\dfrac{1}{n}$ y $P(\mathbb{X}_n=0)=1-\dfrac{1}{n}$\\\\
La pregunta es, ¿a dónde converge en probabilidad $\mathbb{X}_n$?\\\\
Podemos afirmar que $\mathbb{X}_n\overbrace{\longrightarrow}^P 0$ ya que:\\\\
$P ( | \mathbb{X}_n - 0 | \geq\epsilon)=P(| \mathbb{X}_n | \geq\epsilon)=P(| \mathbb{X}_n | =1)=\dfrac{1}{n}\rightarrow 0$
\subsection{Propiedades}
\begin{enumerate}
\item $\mathbb{X}_n\overbrace{\longrightarrow}^P \mathbb{X} \bigwedge \varphi_n\overbrace{\longrightarrow}^P \varphi \Longrightarrow \mathbb{X}_n + \varphi_n\overbrace{\longrightarrow}^P\mathbb{X}+\varphi$
\item $\mathbb{X}_n\overbrace{\longrightarrow}^P \mathbb{X}\Longrightarrow K\mathbb{X}_n\overbrace{\longrightarrow}^P K\mathbb{X}$
\item $\mathbb{X}_n\overbrace{\longrightarrow}^P K$ (es decir, que degenera), entonces ${\mathbb{X}_n}^2\overbrace{\longrightarrow}^P K^2$.\\\\
Para demostrarlo basta notar que: ${\mathbb{X}_n}^2-K^2=(\mathbb{X}_n + K)(\mathbb{X}_n -K)$
\item $\mathbb{X}_n\overbrace{\longrightarrow}^P a \bigwedge \mathbb{X}_n\overbrace{\longrightarrow}^P b\Longrightarrow \mathbb{X}_n \mathbb{\varphi}_n \overbrace{\longrightarrow}^P a\cdot b$
\\\\
Para demostrarlo tenemos que notar que: 
$$\mathbb{X}_n\varphi_n=\dfrac{(\mathbb{X}_n + \varphi_n)^2-\mathbb{X}_n - \varphi_n)^2}{4}\overbrace{\longrightarrow}^P \dfrac{(a+b)^2-(a-b)^2}{4}=ab$$
\item $\mathbb{X}_n\overbrace{\longrightarrow}^P 1 \Longrightarrow \dfrac{1}{\mathbb{X}_n}\overbrace{\longrightarrow}^P 1$
\item $\mathbb{X}_n\overbrace{\longrightarrow}^P a \bigwedge \varphi_n\overbrace{\longrightarrow}^P b\Longrightarrow \mathbb{X}_n\varphi_n^{-1}\overbrace{\longrightarrow}^P ab^{-1}$
\item $\mathbb{X}_n\overbrace{\longrightarrow}^P \mathbb{X}\bigwedge \varphi_n\overbrace{\longrightarrow}^P \varphi \Longrightarrow\mathbb{X}_n\varphi_n\overbrace{\longrightarrow}^P\mathbb{X}\varphi$
\\\\
Para demostrarlo basta notar que: $(\mathbb{X}_n - \mathbb{X})(\varphi_n - \varphi)\overbrace{\longrightarrow}^P 0$ y luego $\mathbb{X}_n\varphi_n-\mathbb{X}\varphi_n-\mathbb{X}_n\varphi-\mathbb{X}\varphi\overbrace{\longrightarrow}^P 0$ (Por la propiedad siguiente)
\item $\mathbb{X}_n\overbrace{\longrightarrow}^P \mathbb{X}$, entonces $\varphi\mathbb{X}_n\overbrace{\longrightarrow}^P \varphi\mathbb{X}$
\end{enumerate}
\section{Funciones características y funciones de distribución}
\section{Leyes de la probabilidad y tipos de leyes}
\end{document}
